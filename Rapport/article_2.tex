%%%%%%%%%%%%%%%%%%%%%%%%%%%%%%%%%%%%%%%%%
% Journal Article
% LaTeX Template
% Version 1.3 (9/9/13)
%
% This template has been downloaded from:
% http://www.LaTeXTemplates.com
%
% Original author:
% Frits Wenneker (http://www.howtotex.com)
%
% License:
% CC BY-NC-SA 3.0 (http://creativecommons.org/licenses/by-nc-sa/3.0/)
%
%%%%%%%%%%%%%%%%%%%%%%%%%%%%%%%%%%%%%%%%%

%----------------------------------------------------------------------------------------
%	PACKAGES AND OTHER DOCUMENT CONFIGURATIONS
%----------------------------------------------------------------------------------------

\documentclass[twoside]{article}

\usepackage{lipsum} % Package to generate dummy text throughout this template

\usepackage[sc]{mathpazo} % Use the Palatino font
\usepackage[T1]{fontenc} % Use 8-bit encoding that has 256 glyphs
\linespread{1.05} % Line spacing - Palatino needs more space between lines
\usepackage{microtype} % Slightly tweak font spacing for aesthetics

\usepackage[hmarginratio=1:1,top=32mm,columnsep=20pt]{geometry} % Document margins
\usepackage{multicol} % Used for the two-column layout of the document
\usepackage[hang, small,labelfont=bf,up,textfont=it,up]{caption} % Custom captions under/above floats in tables or figures
\usepackage{booktabs} % Horizontal rules in tables
\usepackage{float} % Required for tables and figures in the multi-column environment - they need to be placed in specific locations with the [H] (e.g. \begin{table}[H])
\usepackage{hyperref} % For hyperlinks in the PDF

\usepackage{lettrine} % The lettrine is the first enlarged letter at the beginning of the text
\usepackage{paralist} % Used for the compactitem environment which makes bullet points with less space between them

\usepackage{abstract} % Allows abstract customization
\renewcommand{\abstractnamefont}{\normalfont\bfseries} % Set the "Abstract" text to bold
\renewcommand{\abstracttextfont}{\normalfont\small\itshape} % Set the abstract itself to small italic text

\usepackage{titlesec} % Allows customization of titles
\renewcommand\thesection{\Roman{section}} % Roman numerals for the sections
\renewcommand\thesubsection{\Roman{subsection}} % Roman numerals for subsections
\titleformat{\section}[block]{\large\scshape\centering}{\thesection.}{1em}{} % Change the look of the section titles
\titleformat{\subsection}[block]{\large}{\thesubsection.}{1em}{} % Change the look of the section titles

\usepackage{fancyhdr} % Headers and footers
\pagestyle{fancy} % All pages have headers and footers
\fancyhead{} % Blank out the default header
\fancyfoot{} % Blank out the default footer
\fancyhead[C]{Running title $\bullet$ November 2012 $\bullet$ Vol. XXI, No. 1} % Custom header text
\fancyfoot[RO,LE]{\thepage} % Custom footer text

%----------------------------------------------------------------------------------------
%	TITLE SECTION
%----------------------------------------------------------------------------------------

\title{\vspace{-15mm}\fontsize{24pt}{10pt}\selectfont\textbf{Article Title}} % Article title

\author{
\large
\textsc{John Smith}\thanks{A thank you or further information}\\[2mm] % Your name
\normalsize University of California \\ % Your institution
\normalsize \href{mailto:john@smith.com}{john@smith.com} % Your email address
\vspace{-5mm}
}
\date{}

%----------------------------------------------------------------------------------------

\begin{document}

\maketitle % Insert title

\thispagestyle{fancy} % All pages have headers and footers

%----------------------------------------------------------------------------------------
%	ABSTRACT
%----------------------------------------------------------------------------------------

\begin{abstract}

\noindent \lipsum[1] % Dummy abstract text

\end{abstract}

%----------------------------------------------------------------------------------------
%	ARTICLE CONTENTS
%----------------------------------------------------------------------------------------

\begin{multicols}{2} % Two-column layout throughout the main article text

\section{Introduction}

\lettrine[nindent=0em,lines=3]{L} orem ipsum dolor sit amet, consectetur adipiscing elit.
\lipsum[2-3] % Dummy text

%------------------------------------------------

\section{Methods}

Maecenas sed ultricies felis. Sed imperdiet dictum arcu a egestas. 
\begin{compactitem}
\item Donec dolor arcu, rutrum id molestie in, viverra sed diam
\item Curabitur feugiat
\item turpis sed auctor facilisis
\item arcu eros accumsan lorem, at posuere mi diam sit amet tortor
\item Fusce fermentum, mi sit amet euismod rutrum
\item sem lorem molestie diam, iaculis aliquet sapien tortor non nisi
\item Pellentesque bibendum pretium aliquet
\end{compactitem}
\lipsum[4] % Dummy text

%------------------------------------------------



\input{sections\ProjectResults}
\chapter{Conclusion}
This master's thesis seek to investigate some of the common approaches used for \ab{NILM}, in order to determine the capabilities and limits. The SmartHG and the \ab{ECO} dataset have been used in order to test some of the hypothesis raised in the study. It is shown that the quality of data collected form a smart meter over a lossy network is high. The greatest quality diminishing factors in the setup is shown to be malfunctioning equipment. Either on the reviving server or at the equipment in the house. It was investigated if a gap filling approach could improve the quality. Small errors on less than 5 samples is fairly easy to correct. Larger gaps is harder to correct. When using the corrected data in appliance recognition algorithms it is shown that more advanced methods of gap filling does not improve the performance compared to a simple linear interpolation. That said does the linear interpolation improve the performance, in comparison to no effort of correction. 

It is shown that devices that have a relative high consumption is easier to detect than appliances that does only consume a little amount of energy. This is due to the greater chance of uniqueness for these types of appliances. The \df{completeness} and \df{complexity} of statistical disaggregation models where investigated. This showed that one of the factors that had a big impact on the performance was the number of devices in the environment. This was due to \df{appliance interference} between the devices. One way of minimizing the impact of \df{appliance interference} is to look at each phase individually, to minimize the number of appliances in each environment. 

In general is a acceptable performance only obtained on the top consumers in the home. Therefore can the relative consumption of a appliance in compensation to the whole house consumption be used as a quality metric. In the report is a small case project illustrating how a \df{service provider} would be able to sell information about the usage behaviour of televisions in a city. It is shown that for this type of application is the performance acceptable, where detecting small devices like lamps and stereos is way more error prone. 

The \ab{NILM} disaggregation algorithms based on machine learning have a training period prior to deployment. In the training period is statistical disaggregation models learned, that makes the algorithms capable of appliance disaggregation. It is showed that the best performance is achieved when learning in the deployment environment. This is since model learned in a other environment is missing information about the \df{background consumption} in the deployment environment. 

The \ab{NILM} technology seems limited by the low sample rates expected from a smart grid. It is shown that the faster the sample rate, the better the performance. On approach that showed  promise in improving the performance is the \df{norm filter}. The \df{norm filter} enforces the disaggregation with the rules of normal expected use of a appliance. This filters unrealistic event from the disaggregation.
%----------------------------------------------------------------------------------------
%	REFERENCE LIST
%----------------------------------------------------------------------------------------

\begin{thebibliography}{99} % Bibliography - this is intentionally simple in this template

\bibitem[Figueredo and Wolf, 2009]{Figueredo:2009dg}
Figueredo, A.~J. and Wolf, P. S.~A. (2009).
\newblock Assortative pairing and life history strategy - a cross-cultural
  study.
\newblock {\em Human Nature}, 20:317--330.
 
\end{thebibliography}

%----------------------------------------------------------------------------------------

\end{multicols}

\end{document}
