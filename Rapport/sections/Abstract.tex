%Abstract

\begin{abstract}
This master's thesis seek to investigate some of the common approaches used for \ab{NILM}, in order to determine the capabilities and limits. It is assumed that the \ab{NILM} application is deployed in a modern smart grid. The impact the smart grid infrastructure have on the data quality is investigated. It is shown that equipment and network errors are the major quality decreasing factors. Simple gap filling methods are evaluated in order to improve the quality. It is shown that simple gap filling can improve the performance. The performance is also shown to be depended on the number of appliances that are in a given environment, and the consumption requirements of the appliance. Interference from other appliances have a major effect on the performance. Higher sample rates and norm filters is shown decrease the interference. In general is a acceptable performance only obtained on the top consumers in the home. 
\end{abstract}


%Danish Abstract

\renewcommand{\abstractname}{Resumé}
\begin{abstract}
Dette speciale undersøger mulighederne og begrænsningerne med \ab{NILM}. Kvaliteten der kan forventes af data modtaget i en moderne smart grid infrastructure undersøges. Det er vist at fejl i måleudstyr og servere er nogle af de store kvalitets dæmpende aspekter. Forskellige ”Gap filling” metoder er undersøgt, for at udbedre fejlene, og øge kvaliteten. Det er vist at simple ”Gap filling” metoder kan øge kvaliteten. Det er vist at resultatet af \ab{NILM} bliver væsentligt forringet hvis der er mange apparater i det samme miljø. Energikravene for de enkelte apparater har også meget at sige for resultatet. Interferens fra andre apparater er med til at forringe resultatet af \ab{NILM}. Det er vist at højere sample hastigheder vil kunne og ”norm filtre” vil kunne forbedre resultaterne markant. Generelt er det kun de apparater der bruger mest energi i husstanden, der får acceptable resultater. 
\end{abstract}