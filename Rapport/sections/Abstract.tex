%Abstract

\begin{abstract}
This master's thesis seek to investigate some of the common approaches used for \ab{NILM}, in order to determine the capabilities and limits. It is assumed that the \ab{NILM} application is deployed in a modern smart grid. The impact the smart grid infrastructure have on the data quality is investigated. It is shown that equipment and network errors are the major quality decreasing factors. Simple gap filling methods are evaluated in order to improve the quality. It is shown that simple gap filling can improve the performance. The performance is also shown to be depended on the number of appliances that are in a given environment, and the consumption requirements of the appliance. Interference from other appliances have a major effect on the performance. Higher sample rates and norm filters is shown decrease the interference. In general is an acceptable performance only obtained on the top consumers in the household. A small case study illustrates how a \textit{service provider} in a smart grid can benefit from the information collected with a \ab{NILM} application.
\end{abstract}


%Danish Abstract

\renewcommand{\abstractname}{Resumé}
\begin{abstract}
Dette speciale undersøger mulighederne og begrænsningerne med \ab{NILM}. Kvaliteten der kan forventes af data modtaget i en moderne smart grid infrastructure undersøges. Det er vist at fejl i måleudstyr og servere er nogle af de store kvalitets dæmpende aspekter. Forskellige ”Gap filling” metoder er undersøgt, for at udbedre fejlene, og øge kvaliteten. Det er vist at simple ”Gap filling” metoder kan øge kvaliteten. Det er vist at resultatet af \ab{NILM} bliver væsentligt forringet hvis der er mange apparater i det samme miljø. Energikravene for de enkelte apparater har også meget at sige for resultatet. Interferens fra andre apparater er med til at forringe resultatet af \ab{NILM}. Det er vist at højere sample hastigheder vil kunne og ”norm filtre” vil kunne forbedre resultaterne markant. Generelt er det kun de apparater der bruger mest energi i husstanden, der får acceptable resultater. Et lille brugsscenarie viser hvordan en \textit{service provider} kan benytte informationer fra \ab{NILM} applikationer til at udvide deres forretningsmuligheder. 


\end{abstract}

\chapter*{Preface}
This is a master's thesis of M.Sc in computer technology by Rune Arbjerg Heick. The work and writing to the thesis is conducted in the period September 2015 to April 2016. The theme of capability and limits of non-intrusive load monitoring is found in collaboration with Rune
Hylsberg Jacobsen and Emad Samuel Malki Ebeid. It is the hope that techniques and approaches investigated in this work can help support their continued research in the field.

\section*{Reader's Guide}
Through the report, references are made according to the IEEE standard. A bibliography is added at the end of the report. All citations are numbered in the order they are mentioned. Figures, tables and equations are numbered corresponding to the chapter and a sequential number indicating the order in the chapter E.g. Figure 1.3 is the third figure in chapter one.  

All abbreviations is spelled out at first occurrence in each chapter. A list of abbreviations is also included in the beginning of the report. All special definitions is marked with quotation marks at first occurrence in the report. A small description is given at first occurrence, and in the definitions overview in the beginning of the report. All other occurrences is illustrated by italic. Quotation marks is also used at direct and indirect citations or as naming illustrator. 

All appendixes supplied with this report is digital and a link to the online location can be found in the appendix overview, at the end of the report.  

\section*{Acknowledgments}
A special thanks to Rune Hylsberg Jacobsen and Emad Samuel Malki Ebeid for taking the time of giving feedback and improve the thesis project. Thanks to the SmartHG project for allowing access to the Danish SmartHG dataset and thanks to ETH Zurich for providing the ECO dataset. 